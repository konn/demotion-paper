% !TeX root = demotion-paper
\documentclass[demotion-paper]{subfiles}
\begin{document}
\section{Disjunctive Constraints}
\label{sec:disj}
It is sporadically complained that type-classes in Haskell lacks \emph{disjunction}, or \emph{logical-or} in its constraint language.
As for general type-classes, excluding disjunction from the type-classes language is a rational design decision for several reasons.
To name a few:
\begin{enumerate}
  \item Type-classes adopts open-world hypothesis: users can add new instances freely, so the result of the instance resolution can differ in context.
  \label{item:openness}
  \item The semantics is not clear when multiple disjunctive clauses can be satisfied simultaneously.
  \label{item:arb-choice}
\end{enumerate}
In some cases, however, the above obstacles can be ignored, and in most cases, unrestricted disjunctions are unnecessary.
For example, when one wants to switch implementations of instances based on particular shapes of a constructor, it is customary to use \hask{{-# OVERLAPPABLE #-}} or \hask{{-# INCOHERENT #-}} pragmas.
But, in some cases, this makes instance resolution unpredictable and incoherent (as indicated by its name), and sometimes doesn't work well with advanced type hackery.

In this section, we propose another way of emulating disjunctive constraints, applying the constructive point of view.
First, we recall the ``meaning'' of disjunction $\varphi \vee \psi$ in BHK interpretation:
\[
  A: \varphi \vee \psi \iff A = \braket{i, B},\text{ where }
  i = 0 \text{ and } B: \varphi, \text{ or }
  i = 1 \text{ and } B: \psi
\]
That is, a ``witness'' $A$ of $\varphi \vee \psi$ is a tagged union of ``witnesses'' of $\varphi$ and $\psi$.

It suggests that if constraints of interest can be expressed by or recovered from \emph{witnesses}, we can take their disjunction by choosing one of such witnesses.
If such witnesses can be computed statically and deterministically, the obstacle \Cref{item:openness} is not a problem.
Further, if users can explicitly \emph{manipulate} such witnesses concretely, we can control the selection strategy of disjunctive clauses, enabling us to resolve \Cref{item:arb-choice} manually.
\end{document}
