%!TeX root = demotion-paper.tex
\documentclass[demotion-paper.tex]{subfiles}
\begin{document}
\section{Disjunctive Constraints}
\label{sec:disj}
It is sporadically complained that type-classes in Haskell lacks \emph{disjunction}, or \emph{logical-or} in its constraint language.
As for general type-classes, excluding disjunction from the type-classes language is a rational design decision for several reasons.
To name a few:
\begin{enumerate}
  \item Type-classes adopts open-world hypothesis: users can add new instances freely, so the result of the instance resolution can differ in context.
  \label{item:openness}
  \item The semantics is not clear when multiple disjunctive clauses can be satisfied simultaneously.
  \label{item:arb-choice}
\end{enumerate}
In some cases, however, the above obstacles can be ignored, and in most cases, unrestricted disjunctions are unnecessary.
For example, when one wants to switch implementations of instances based on particular shapes of a constructor, it is customary to use \hask{{-# OVERLAPPABLE #-}} or \hask{{-# INCOHERENT #-}} pragmas.
But, in some cases, this makes instance resolution unpredictable and incoherent (as indicated by its name), and sometimes doesn't work well with advanced type hackery.

In this section, we propose another way of emulating disjunctive constraints, applying the constructive point of view.
First, we recall the ``meaning'' of disjunction $\varphi \vee \psi$ in BHK interpretation:
\[
  A: \varphi \vee \psi \iff A = \braket{i, B},\text{ where }
  i = 0 \text{ and } B: \varphi, \text{ or }
  i = 1 \text{ and } B: \psi
\]
That is, a ``witness'' $A$ of $\varphi \vee \psi$ is a tagged union of ``witnesses'' of $\varphi$ and $\psi$.

It suggests that if constraints of interest can be expressed by or recovered from \emph{witnesses}, we can take their disjunction by choosing one of such witnesses.
If such witnesses can be computed statically and deterministically, the obstacle \Cref{item:openness} is not a problem.
Further, if users can explicitly \emph{manipulate} such witnesses concretely, we can control the selection strategy of disjunctive clauses, enabling us to resolve \Cref{item:arb-choice} manually.

\subsection{Motivating Example: Field Extraction of Union Types}
OK, let us see the example.
Here, we consider the following \hask{HasFactor}-class:

\begin{code}
class HasFactor a b where
  getFactor :: b -> a
\end{code}

The intended semantics of \hask{HasFactor a b} is that ``a type \hask{b} has at least one field of type \hask{a}'', and \hask{getFactor} gives such a value.
In short, the problem we want to tackle in this section is as follows:
\begin{quote}
  \emph{How can we give a} \hask{HasFactor a}\emph{-instance for union types}?
\end{quote}

\subsubsection{Extensible Records}
To make the situation simpler, let us introduce another player into the scene: the type \hask{Record h ks} of \emph{extensible records}.
That is, a value of type \hask{Record h ks} is a record-like structure isomorphic to a record with field labels \hask{ks :: [key]}, where each label \hask{k} in \hask{ks} is associated to the value of type \hask{h k}:

\begin{code}
data Record f keys where
  EmptyRecord :: Record f '[]
  (:<) :: f k -> Record f ks -> Record f (k ': ks)
\end{code}

That is, we implement \hask{Record} as a heterogeneous linked-list\footnote{%
This implementation is not meant to be practical.
In practice, it is much more convenient to allow specify labels and corresponding field type independently, i.e.\ parametrise \hask{Record} over \hask{h :: key -> val -> Type}.
And if the efficiency of field access matters, underlying representation of extensible records must be arrays or vectors with $O(1)$ random access.
Furthermore, it is desirable to use type-level ordered maps instead of lists to tune-up type-cheking speed.
Several efficient implementations can be found on Hackage~\cite{Kinoshita:2020aa,Sterling:2020aa,Thiemann:2020aa}, and we are also planning to publish the in-house packge (still in-progress, though).
Here, we adopt the above simple API for an exposititory purpose.}.
We also need a field accessor to retrieve information from a record:
\begin{code}
data Index k ks where
  Here  :: Index k (k ': ks)
  There :: Index k ks -> Index k (k' ': ks)

-- ...Obvious singleton SIndex and their Sing/Known instances here...

walkIndex :: Index k ks -> Record f ks -> f k
walkIndex Here (v :< _) = v
walkIndex (There trail) (_ :< rest) = walkIndex trail rest  
\end{code}
A type \hask{Index k ks} is a \emph{witness} of the membership of label \hask{k} in a type-level list \hask{ks}, and \hask{walkIndex} walks an extensible records along a given \hask{Index k ks} and retrieves a value of \hask{h k}\footnote{The same remark on efficiency as above also applies here: it is much more practical to use an index number represented as a newtyped \hask{Int} if one needs $O(1)$ random access on fields.
In such a case, one has to use \hask{unsafeCoerce} carefully to convince the compiler.}.

We can also compute \hask{Index} at type-level in an obvious way\footnote{Oh, you noticed something? Well, in \Cref{sec:plugins}, we will turn to the implementation of \hask{FindIndex} again.}, where \hask{(<$>)} is a type-level analogue of \hask{fmap}:
\begin{code}
type family FindIndex k ks where
  FindIndex _ '[] = 'Nothing
  FindIndex k (k ': ks) = 'Just 'Here
  FindIndex k (_ ': ks) = 'There <$> FindIndex k ks  

type FindIndex' k ks = FromJust ('Text "key not found") (FindIndex k ks)
\end{code}
\hask{FindIndex'} is a variant of \hask{FindIndex}, which returns raw membership at type-level if it is found, and throwing a type-error otherwise.

Now that we can compute the index of labels statically, one can implement a variant of \hask{walkIndex} with the label-index statically inferred:
\begin{code}
type Member k ks = Known (FindIndex' k ks)
getRecField :: forall keks h. Member k ks
  => Record h ks -> h k
getRecField = walkIndex $ demote $ sing @(FindIndex' k ks)
\end{code}
An idea is that if the concrete value of \hask{FindIndex' k ks} is \hask{Known} at the compile-time, we can use it to retrieve a field in a record.
Note that if the value of \hask{FindIndex key keys} was \hask{'Nothing}, \hask{FindIndex' key keys} reduces to a type-error.
In such a case, since the type-level language of Haskell is strict, the entire constraint \hask{Known (FindIndex' key keys)} throws a type-error during instance resolution:
\begin{repl}
>>> getRecField @Bool (EmptyRecord :: Record Maybe '[])
Key `Bool' is absent in the list: '[]

>>> getRecField @Bool (Just 'a' :< Just True :< Nothing @() :< EmptyRecord)
Just True
\end{repl}

Now, we can give an implementation of \hask{HasFactor (h a) (Record h keys)} with these equipments:
\begin{code}
instance (Member a keys, h ~ h') => HasFactor (h' a) (Record h keys) where
  getFactor = getRecField
\end{code}

\subsubsection{Factor of a Union --- singleton strikes again!}
Now we enter the situation where we want disjunctive constraints.
Consider the following type paring two records together:
\begin{code}
data UnionedRecord (h :: k -> Type) (ls :: [k]) (rs :: [k]) = 
  UnionRec { recL :: Record h ls, recR :: Record h rs }
\end{code}
Now, how can we implement instances like \hask{HasFactor (h a) (UnionedRecord h ls rs)}?
Na\"{i}vely, one might at first want to write an instance like:
\begin{code}
instance (HasFactor (h a) (Record h ls) `Or` HasFactor (h a) (Record h rs))
      => HasFactor (h a) (UnionedRecord h ls rs) where ...
\end{code}
But, as noted in the first section, there is no such things in Haskell as restricted \hask{Or}-constraint.

Fortunately, our \hask{HasFactor (h a) (Record h ls)} instance is \emph{witnessed} by a concrete type: \hask{Index a ls}!
Since its concrete value can be computed statically by \hask{FindIndex} type-family, why not combining the results of them\footnote{Here, \hask{(<|>)} is a type-level left-biased choice operator on \hask{Maybe}s.}?
\begin{code}
type UnionedIndex' k ls rs =
  FromJust ('Text "No label found in both key")
    ('Left <$> FindIndex k ls <|> 'Right <$> FindIndex k rs)

instance
  Known (UnionedIndex' a ls rs) =>
  HasFactor (h a) (UnionedRecord h ls rs) where
  getFactor (UnionRec l r) =
    case sing @(UnionedIndex' a ls rs) of
      SLeft pth -> withKnown pth $ getFactor l
      SRight pth -> withKnown pth $ getFactor r
\end{code}
Everything seems fine, but then GHC complains on call-sites of \hask{getFactor}:
\begin{repl}
• Could not deduce (Known (FromJust ('Text "key not found") (FindIndex a ls)))
  arising from a use of ‘getFactor’
  from the context: Known (UnionedIndex' a ls rs)
...
• In the second argument of ‘($)’, namely ‘getFactor l’
  In the expression: withKnown pth $ getFactor l

... similar error for `getFactor r' ...
\end{repl}
The error seems weired at first glance: we are just giving the \hask{Known} dictionary with \hask{withKnown pth}, where \hask{pth} corresponds either \hask{Index a ls} or \hask{Index a rs}. Why?

The root cause of this error is that the type-checker doesn't know the following facts:
\begin{enumerate}
  \item \hask{FromJust ('Text ...)} commutes with \hask{(<|>)}, and
  \item If \hask{FromJust msg ma} reduces, then \hask{ma ~ 'Just (FromJust msg' ma)} for any \hask{msg'}.
\end{enumerate}
Although these two facts seems rather obvious, it needs some non-trivial axioms to infer.
Hence, if we want to convince the compiler without modifying the instance definitions, we have to augment the compiler with type-checker plugins~\cite{GHC-Team:2020aa}.
Although writing type-checker plugins is a fun, but not so easy to implement it \emph{correctly}. Is there any other way to avoid this obstacle?

At this point, we must notice one fact: we can still use \emph{any} value of type \hask{Index k ks} to retrieve a value of type \hask{h k} from \hask{Record h ks};.Indeed, as \hask{ks} can have duplicated elements, there can be more than one distinct \hask{Index k ks} at the same time, e.g.\ for \hask{Index 3 '[3 ,5, 4, 3]}.
\hask{FindIndex} was just a canonical way of computing the left-most such index, if present.
To summary, requiring \hask{Known (LookupIndex' k ks)} in \hask{getRecField} and the \hask{HasFactor}-instance for extensible records was just too much.
What we really need is a constraint demanding ``there is at least one value of type \hask{Index k ks} given'', embodied by the following class and helper functions:
\begin{code}
class Given a where
  given :: a

give :: a -> (Given a => r) -> r
\end{code}
These are excerpted from widely-used \texttt{reflection} package~\cite{Kmett:2020aa} which implements Implicit Configuration~\cite{Kiselyov:2004aa}.

With this, we can rewrite \hask{getRecField} and the corresponding \hask{HasFactor}-instance in more robust way:
\begin{code}
type Member k ks = Given (Index k ks)

getRecField :: forall key keys h. Member key keys => Record h keys -> h key
getRecField = walkIndex given  -- changed from @sing@

-- | Serves as a default instance for @Member@.
-- Can be safely overriden by 'give' operator.
instance Known (FindIndex' k ks)
      => Given (Index k ks) where
  given = demote $ sing @(FindIndex' k ks)

instance (Given (Index a keys), h ~ h')
      => HasFactor (h' a) (Record h keys) where
  getFactor = getRecField
\end{code}

With this, we can now successfully implement \hask{HasFactor}-instance for \hask{UnionedRecord} as follows:

\begin{code}
-- Just to avoid pollution by orphan @Given@ instance
newtype IndexUnion k ls rs = WrapIndexUnion
  {getUnionedIndex :: Either (Index k ls) (Index k rs)}

instance Known (UnionedIndex' k ls rs) => Given (IndexUnion k ls rs) where
  given = WrapIndexUnion $ demote $ sing @(UnionedIndex' k ls rs)

instance Given (IndexUnion k ls rs)
      => HasFactor (h k) (UnionedRecord h ls rs) where
  getFactor (UnionRec l r) =
    case getUnionedIndex $ given @(IndexUnion k ls rs) of
      Left pth -> give pth $ getFactor l
      Right pth -> give pth $ getFactor r
\end{code}

In above two implementations, \hask{Given} instance serves as a ``default instance'' to calculate witness.
As already shown in the implementation of \hask{getFactor} for \hask{UnionRec}, it can be overriden by \hask{give} operator without resulting in type errors complaining about overlapping instance or ``Could not deduce (Known ...)''.
This flexibility is the reason we chose to use \hask{Given} class instead of \texttt{ImplicitParams} GHC extention providing similar functionality of dynamic scoping:  the instance shadowing in \texttt{ImplicitParams} can have unpredictable behaviour.

Let us check it works as expected:

\begin{repl}
>>> theRec = Const "Hehe" :< Const "Foo" :< Const "buz" :< EmptyRecord
      :: Record (Const String) '[5,42, 34]
>>> anotherRec = Const "Phew" :< Const "Wow" :< Const "Cool" :< EmptyRecord
      :: Record (Const String) '[94, 5, 2]
>>> getFactor @(Const String 42) (UnionRec theRec anotherRec)
Const "Foo"

>>> getFactor @(Const String 2) (UnionRec theRec anotherRec)
Const "Cool"

>>> getFactor @(Const String 5) (UnionRec theRec anotherRec)
Const "Hehe"

-- Beware of reordering:
>>> getFactor @(Const String 5) (UnionRec anotherRec theRec)
Const "Wow"

>>> getFactor @(Const String 999) (UnionRec theRec anotherRec)
A field of type `999' not found in either of fields:
   Left: '[5, 42, 34]
  Right: '[94, 5, 2]  
\end{repl}

\end{document}
