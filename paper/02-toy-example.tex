%!TeX root = demotion-paper.tex
\documentclass[demotion-paper.tex]{subfiles}
\begin{document}
\section{Toy Example: Demoting Type-level GCD}
\label{sec:gcd}
Let us begin with a simple example of type-level greatest common divisors (GCDs):

\begin{code}
import GHC.TypeLits

type family GCD n m where
  GCD 0 m = m
  GCD n 0 = n
  GCD n m = GCD (Mod m n) n
\end{code}

So far, so good.

\begin{repl}
>>> :kind! GCD 12 9
GCD 12 9 :: Nat
= 3
\end{repl}

Suppose we want to ``demote'' this definition of \hask{GCD} to expression-level using singletons, that is, to implement the following function \hask{sGCD}:

\begin{code}
sGCD :: SNat n -> SNat m -> SNat (GCD n m)
\end{code}

First, we need to test the equality of type-level naturals.
In the \texttt{base} package, there is a suitable type-classs for it:
\begin{code}
-- Defined in Data.Type.Equality in base
class TestEquality f where
  testEquality :: f a -> f b -> Maybe (a :~: b)

data (:~:) a b where Refl :: a :~: a
\end{code}
Assuming the \hask{TestEquality SNat} instance, one might first attempt to write it as follows:
\begin{code}
sGCD :: SNat n -> SNat m -> SNat (GCD n m)
sGCD sn sm =
  case ( testEquality sn (sNat @0), 
         testEquality sm (sNat @0)) of
    (Just Refl, _) -> sm
    (_, Just Refl) -> sn
    (Nothing, Nothing) -> sGCD (sMod sm sn) sn
\end{code}

The first two cases type-check as expected, but the last case results in the following type error:

\begin{repl}
• Couldn't match type ‘GCD (Mod m n) n’
  with ‘GCD n m’
  Expected type: SNat (GCD n m)
    Actual type: SNat (GCD (Mod m n) n)
  NB: ‘GCD’ is a non-injective type family...
\end{repl}

Why? The definition of \hask{sGCD} seems almost literally the same as type-level \hask{GCD}.
It first match \hask{n} against \hask{0}, then \hask{m} against \hask{0}, and finally fallbacks to \hask{GCD (Mod m n) n}.

Carefully analysing the first two cases, one can realise that there are additional type-level constraints introduced by \hask{Refl} GADT constructor:

\begin{code}
  Refl :: a ~ b => a :~: b
\end{code}

Thus, in the first two cases, the compiler can tell either \hask{n ~ 0} or \hask{m ~ 0}.
Since the \hask{GCD} is defined as a closed type family, the compiler can match clauses in a top-down manner and successfully apply either of the first two clauses of the definition of \hask{GCD}.
In other words, the constructor \hask{Refl} \emph{witnesses} the evaluation path of type-level function \hask{GCD} in the first two cases.

In the last case, however, no additional type-level constraint is available.
Although humans can still think ``as all the \hask{Refl} clauses failed to match, hence the non-equal clause must apply here'', this intuition is not fully expressed in the type-level constraint!

So we have to give the compiler some \emph{witness} also in the last case.
What kind of a witness is needed here?
Well, we need to teach the compiler which clause was actually used.
In this case, branching is caused by type-level equality: the evaluation path depends on whether \hask{n} or \hask{m} is \hask{0} or not.
First, let us make this intuition clear in the definition of \hask{GCD}:

\begin{code}
import Data.Type.Equality (type (==)) -- from base

type GCD n m = GCD_ (n == 0) (m == 0) n m
type family GCD_ nEq0 mEq0 n m :: Nat where
  GCD_ 'True  _      _ m = m  -- n ~ 0; return m
  GCD_ 'False 'True  n _ = n  -- m ~ 0; return n
  GCD_ 'False 'False n m =    -- Neither; recur!
    GCD_ (Mod m n == 0) 'False (Mod m n) n
\end{code}

Here, we have two type-level functions: newly defined one, \hask{GCD_}, is the main loop implementing Euclidean algorithm, and
\hask{GCD} is redefined to call \hask{GCD_} with the needed information.
Now, \hask{GCD_} takes not only natural numbers but also a type-level \hask{Bool}s \emph{witnessing} equality of \hask{n} and \hask{m} with \hask{0}.
From this, GHC can tell which clause is taken from the first two type-arguments.
As clauses in closed type families can be viewed as a mutually exclusive alternatives, this approach shares the spirit with the constructive BHKs interpretation of $\vee$.

Now that we can give the compiler witnesses as the first two type-arguments of \hask{GCD_}, we are set to implement \hask{sGCD}.
First, we need \emph{demoted} version of type-level \hask{(==)}.
The first attempt might go as follows:
\begin{code*}{escapeinside=||,linenos,gobble=0}
(%==) :: TestEquality f
      => f a -> f b -> SBool (a == b)
sa %== sb = case testEquality sa sb of
  Just Refl -> STrue |\label{line:trueclause}|
  Nothing -> SFalse
\end{code*}
Unfortunately, this doesn't work as expected.
The first error on \hask{STrue} (\lref{line:trueclause}) says:

\begin{repl}
• Could not deduce: (a == a) ~ 'True
  from the context: b ~ a
    bound by a pattern with constructor:
               Refl :: forall k (a :: k). a :~: a,
             in a case alternative
    at /.../GCD.hs:33:8-11
  Expected type: SBool (a == b)
    Actual type: SBool 'True
\end{repl}

This is due to the definition of type-level \hask{(==)} in GHC base library:

\begin{code}
type family a == b where
  f a == g b = (f == g) && (a == b)
  a   == a   = 'True
  _   == _   = 'False
\end{code}
As described in the documentation~\cite{GHC-Team:2021aa}, the intuition behind the definition of the first clause is to let the compiler to infer, e.g.\ \hask{Just a == Just b} from \hask{a == b}.

This behaviour is desirable when one treats equalities involving compound types, like \hask{(f a == g b) ~ 'True}.
But when one wants to give a witness of \hask{(a == b) ~ 'True}, we cannot make use of \hask{Refl}.
This is, again, due to the lack of witness of the evaluation path: the compiler cannot determine which clause should be taken to compute \hask{a == b} if \hask{a} and \hask{b} are both opaque variable!

A solution here is just to define another type family, which requires the reflexivity only:
\begin{code}
type family a === b where
  a === a = 'True
  _ === _ = 'False
\end{code}

Although this equality cannot treat equalities between compound types inductively, it suffices for \hask{GCD} case.
We will revisit to a treatment of type-level equality in \Cref{sec:plugins}.
Demoted version of this now gets:

\begin{code*}{escapeinside=||,gobble=0}
(%===) :: TestEquality f
       => f a -> f b -> SBool (a === b)
sa %=== sb = case testEquality sa sb of
  Just Refl -> STrue
  Nothing -> SFalse |\label{line:falseclause}|
\end{code*}

Now, the type-error remains on \lref{line:falseclause}: \hask{SFalse}.
This is, again, due to the lack of witness of being distinct.
But wait! We are just struggling to produce such a negative witness of non-equality, which in turn requires itself. A vicious cycle!
At this very point, there is no other way than resorting to the ancient cursed spell \hask{unsafeCoerce}:
\begin{code}
import Unsafe.Coerce

sa %=== sb = case testEquality sa sb of
  Just Refl -> STrue
  Nothing -> unsafeCoerce SFalse  
\end{code}
This use of \hask{unsafeCoerce} is inherently inevitable.
Fortunately, provided that \hask{TestEquality} instance is implemented soundly, this use of \hask{unsafeCoerce} is not cursed: this is just postulating an axiom that is true but there is no way to tell it to the compiler safely.
If one wants to construct evidence of type-level (non-)equality solely from the expression, we must assume some axiom and introduce it by \hask{unsafeCoerce}.
This is what library builders usually do when they implement basic (expression-level) operators to manipulate type-level values.
Such ``trust me'' axioms can be found, for example, in \hask{TestEquality TypeRep} instance in \texttt{base}, and various \hask{SEq} instances in \texttt{singletons}~\cite{singletons} package\footnote{There is another way of introducing axioms: invoking type-checker plugins.}.
Anyway, we are finally at the point of implementing working \hask{sGCD}, replacing every occurrence of \hask{(==)} with our custom \hask{(===)}:

\begin{code}
type GCD n m = GCD_ (n === 0) (m === 0) n m

type family GCD_ nEq0 mEq0 n m :: Nat where
  GCD_ 'True  _      _ m = m
  GCD_ 'False 'True  n _ = n
  GCD_ 'False 'False n m = 
    GCD_ (Mod m n === 0) 'False (Mod m n) n

sGCD :: SNat n -> SNat m -> SNat (GCD n m)
sGCD sn sm =
  case (sn %=== sZero, sm %=== sZero) of
    (STrue, _) -> sm
    (SFalse, STrue) -> sn
    (SFalse, SFalse) -> sGCD (sMod sm sn) sn
\end{code}

Finally, the compiler is happy with all the definitions!
We can confirm that the above \hask{sGCD} works just as expected:
\begin{repl}
>>> sGCD (sNat @12) (sNat @30)
6
\end{repl}

\subsection{Summary}
When writing closed type-families, it is useful to introduce arguments \emph{witnessing} evaluation paths explicitly, interpreting clauses in a closed family as mutually exclusive alternatives.
It makes it easy to write its demoted singletonised functions.
As there are several variants of type-level equalities, we must choose appropriate one carefully; in the GCD case it is convenient to use the type-level Boolean equality which takes only reflexivity into account.
\end{document}
