%!TeX root = demotion-paper.tex
\documentclass[demotion-paper.tex]{subfiles}
\begin{document}
\section{Toy Example: Demoting Type-level GCD}
\label{sec:gcd}
Let's begin with a simple example of type-level greatest common divisors (GCDs).
It is quite simple as all we have to do is just to implement Euclidean algorithm:

% TODO: Clarify the API and libraries we use

\begin{code}
{-# LANGUAGE DataKinds, TypeFamilies, UndecidableInstances #-}
import GHC.TypeLits

type family GCD (n :: Nat) (m :: Nat) :: Nat where
  GCD 0 m = m
  GCD n 0 = n
  GCD n m = GCD (Mod m n) n
\end{code}

So far so good.

\begin{repl}
>>> :kind! GCD 12 9
GCD 12 9 :: Nat
= 3
\end{repl}

Suppose we want to ``demote'' this definition of ``GCD'' to expression-level using singletons, i.e.\ we want to implement the following function \hask{sGCD}:

\begin{code}
sGCD :: SNat n -> SNat m -> SNat (GCD n m)
\end{code}

Assuming that \hask{SNat} has \hask{TestEquality} instance, one might first attempt to write it as follows:

\begin{code}
sZero :: SNat 0
sZero = sNat

sGCD :: SNat n -> SNat m -> SNat (GCD n m)
sGCD sn sm = case (testEquality sn sZero, testEquality sm sZero) of
  (Just Refl, _) -> sm
  (_, Just Refl) -> sn
  (Nothing, Nothing) -> sGCD (sMod sm sn) sn
\end{code}

First two lines type-checks as expected, but the last case results in the following type error:

\begin{repl}
• Couldn't match type ‘GCD (Mod m n) n’ with ‘GCD n m’
  Expected type: SNat (GCD n m)
    Actual type: SNat (GCD (Mod m n) n)
  NB: ‘GCD’ is a non-injective type family
• In the expression: sGCD (sMod sm sn) sn
  In a case alternative: (Nothing, Nothing) -> sGCD (sMod sm sn) sn
  In the expression:
    case (testEquality sn sZero, testEquality sm sZero) of
      (Just Refl, _) -> sm
      (_, Just Refl) -> sn
      (Nothing, Nothing) -> sGCD (sMod sm sn) sn
\end{repl}

Why? The definition of \hask{sGCD} seems almost literally the same as type-level \hask{GCD}.
It first match \hask{n} against \hask{0}, then \hask{m} against \hask{0}, and finally fallbacks to the term \hask{GCD (Mod m n) n}.

Carefully analysing first two cases, one can realise that there are additional type-level constraint introduced by \hask{Refl} GADT constructor:

\begin{code}
data (:~:) a b where
  Refl :: a :~: a
\end{code}

Thus, in the first two cases, the compiler can tell either \hask{n ~ 0} or \hask{m ~ 0}.
Since the \hask{GCD} is defined as a closed type family, the compiler can match clauses in a top-down manner and successfully apply either of the first two clauses of the definition of \hask{GCD} and type-checker passes.
In other words, the constructor \hask{Refl} \emph{witnesses} the evaluation path of type-level function \hask{GCD}.

In the last case, however, no additional type-level constraint is available.
Despite humans can still think ``as all the \hask{Refl} clauses failed to match, hence the non-equal clause must apply here'', this intuition is not fully expressed in the type-level constraint!

So we have to give compiler some \emph{witness} to convince them.
What kind of witness is needed here?
Well, we need to teach the compiler which clause of the definition was actually used.
In this case, branching is caused by type-level equality: the evaluation path depends on whether \hask{n} or \hask{m} is \hask{0} or not.
First, let's make this intuition clear in the definition of \hask{GCD}:

\begin{code}
import Data.Type.Equality (type (==)) -- from base

type GCD n m = GCD_ (n == 0) (m == 0) n m
type family GCD_ nEq0 mEq0 n m :: Nat where
  GCD_ 'True  _      _ m = m   -- n ~ 0; return m
  GCD_ 'False 'True  n _ = n   -- m ~ 0; return n
  GCD_ 'False 'False n m = GCD_ (Mod m n == 0) 'False (Mod m n) n  -- neither!
\end{code}

Here, we have two type-level functions: newly defined one, \hask{GCD_}, is a main loop implementing Euclidean algorithm.
Now, \hask{GCD_} takes not only natural numbers to calculate \hask{GCD}, but also type-level \hask{Bool}ean-values that record equality of \hask{n} and \hask{m} with \hask{0}.
From this, GHC can tell exactly which clause is taken from the first two type-arguments.
Then, the redefined type synonym \hask{GCD} just calls \hask{GCD_} with the needed information.

Now that we can give the compiler witnesses as the first two type-arguments of \hask{GCD_}, we are set to implement \hask{sGCD}.

First, we need \emph{demoted} version of type-level \hask{(==)}.
The first attempt might go as follows:
\begin{code}
data SBool p where
  STrue  :: SBool 'True
  SFalse :: SBool 'False

(%==) :: TestEquality f => f a -> f b -> SBool (a == b)
sa %== sb = case testEquality sa sb of
  Just Refl -> STrue
  Nothing -> SFalse
\end{code}
Unfortunately, this doesn't work as expected.
The first error on \hask{STrue} says:

\begin{repl}
• Could not deduce: (a == a) ~ 'True
  from the context: b ~ a
    bound by a pattern with constructor:
               Refl :: forall k (a :: k). a :~: a,
             in a case alternative
    at /.../GCD.hs:33:8-11
  Expected type: SBool (a == b)
    Actual type: SBool 'True
\end{repl}

This is due to the definition of type-level \hask{(==)} in GHC base library:

\begin{code}
type family a == b where
  f a == g b = (f == g) && (a == b)
  a   == a   = 'True
  _   == _   = 'False
\end{code}

The complication of the first clause is intensional.
As described in the documentation~\cite{GHC-Team:2021aa}, the intuition behind this is to let the compiler to infer, e.g.\ \hask{Just a == Just b} from \hask{a == b}.

This behaviour is desirable when one \emph{produces} the constraints of forms \hask{(f a == g b) ~ 'True}.
But when one wants to \emph{eliminate} an evidence of \hask{(a == b) ~ 'True}, we cannot make use of the evidence given by \hask{Refl}.
This is, again, due to the lack of witness of the evaluation path: the compiler cannot determine which clause should be taken to compute \hask{a == b} if \hask{a} and \hask{b} are both opaque variable!

A solution here is just to define another type family, which requires only bare equality:
\begin{code}
type family a === b where
  a === a = 'True
  _ === _ = 'False
\end{code}

Although this equality cannot treat equalities between compound types inductively, it suffices for \hask{GCD} case.
We will revisit to a treatment of type-level equality in \Cref{sec:plugins}.
Demoted version of this now gets:

\begin{code}
(%===) :: TestEquality f => f a -> f b -> SBool (a === b)
sa %=== sb = case testEquality sa sb of
  Just Refl -> STrue
  Nothing -> SFalse  
\end{code}

Now, the type-error remains on the last clause: \hask{SFalse}.
This is also due to the lack of witness of being distinct.
But, wait! We are just struggling to produce such negative witnesses of non-equality, which in turn requires itself. A vicious cycle!

At this very point, there is no other way than resorting to the ancient cursed spell: \hask{unsafeCoerce}:

\begin{code}
import Unsafe.Coerce

(%===) :: TestEquality f => f a -> f b -> SBool (a === b)
sa %=== sb = case testEquality sa sb of
  Just Refl -> STrue
  Nothing -> unsafeCoerce SFalse  
\end{code}

This use of \hask{unsafeCoerce} is inherently inevitable.
Fortunately, provided that \hask{TestEquality} instance is implemented soundly, this use of \hask{unsafeCoerce} is not cursed: this is just postulating an axiom that is true but there is no way to tell it to the compiler safely\footnote{Of course, one must pay attension whether coerced values have the same representation. In this case, it is legitimate useage.}.
If one wants to construct evidence of type-level (non-)equality solely from the expression, we must assume some axiom and introduce it by \hask{unsafeCoerce}.
This is how the library builder usually do when they implement basic (expression-level) operators to manipulate type-level values.
Such ``trust me'' axioms can be found, for example, in \hask{TestEquality TypeRep} instance in \texttt{base}, and various \hask{SEq} instances in \texttt{singletons}~\cite{singletons} package.

Anyway, we are finally at the point of implementing working \hask{sGCD}, replacing every occurrence of \hask{(==)} with our custom \hask{(===)}:

\begin{code}
type GCD n m = GCD_ (n === 0) (m === 0) n m

type family GCD_ nEq0 mEq0 n m :: Nat where
  GCD_ 'True  _      _ m = m -- n is zero; return m
  GCD_ 'False 'True  n _ = n -- m is zero; return n
  GCD_ 'False 'False n m = GCD_ (Mod m n === 0) 'False (Mod m n) n

sGCD :: SNat n -> SNat m -> SNat (GCD n m)
sGCD sn sm = case (sn %=== sZero, sm %=== sZero) of
  (STrue, _) -> sm
  (SFalse, STrue) -> sn
  (SFalse, SFalse) -> sGCD (sMod sm sn) sn
\end{code}

Finally, the compiler gets happy with all the definitions!
We can confirm that the above \hask{sGCD} works just as expected:
\begin{repl}
>>> sGCD (sNat @12) (sNat @30)
6
\end{repl}

\end{document}
