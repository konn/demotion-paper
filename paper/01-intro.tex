%!TeX root = demotion-paper.tex
\documentclass[demotion-paper.tex]{subfiles}
\begin{document}
\section{Introduction}
Since Haskell had been given a Promotion~\cite{Yorgey:2012}, using Haskell with dependent types is a joy.
It's not only a joy: it gives us a neat language to express invariants of programs with type-level constraints.
But it also comes with pain: writing correct but maintainable dependently-typed programs in Haskell sometimes is a hard job, as explained by Lindley and McBride~\cite{10.1145/2503778.2503786}.
It is particularly hard when one tries to bridge a gap between types and expressions, to maintain complex type constraints, and so on.

Here, we propose to borrow the wisdom from one of the greatest discovery in Computer Science and Logic: Curry--Howard correspondence.
It tells us that types are propositions and programs are proofs in intuitionistic logic, in a rigorous sense.
According to Brouwer--Heyting--Kolmogorov (BHK) interpretation\footnote{For the precise historical background, we refer readers to the article by Wadler~\cite{Wadler:2015aa}.}, its informal ancestor, a proof of an intuitionistic proposition is some kind of a construction, or \emph{witness}, conveying the information of the proposition.
For example, a proof of $\varphi \land \psi$ is given by a pair of proofs $T: \varphi$ and $S: \psi$; a proof of $\varphi \to \psi$ is given by a function taking a proof of $\varphi$ and returns that of $\psi$, and so on.
The interpretation of our particular interest in this paper is that of disjunction:
\begin{gather*}
  A: \varphi \vee \psi \iff A = \braket{i, B},
\\
\text{where }
i = 0 \text{ and } B: \varphi, \text{ or }
i = 1 \text{ and } B: \psi
\end{gather*}
That is, a witness of disjunction is given by a tuple of the tag recording which case holds and its corresponding witness.
Although BHK interpretation is not as rigorous as Curry--Howard correspondence, its looseness allows us much broader insights, as we shall see in the present paper.

In particular, we propose the following techniques for dependently-typed programming in Haskell:
\begin{enumerate}
\item Augment type-level functions with ``witnesses'' so that we can recover the information of conditional branchings when we want to demote such functions with singletons;
\item Using witnesses carrying the information of instance dictionaries to manipulate type-level constraints statically and dynamically.
\item A design principle for the witness of type-level equality.
\end{enumerate}
These princples are useful when one wants to \emph{demote} type-level machinery down to expression level using \emph{singletons}.
Such a need naturally arises when one already has a dependently-typed API and later one has to resolve type-level constraints dynamically at runtime.
To see these, the paper is organised as follows.
In \Cref{sec:gcd}, we use a type-level GCD as an example to demonstrate how type-level \emph{witnesses} can be used for such a \emph{demotion} of closed type-level functions involving pattern-matchings.
\Cref{sec:disj} demonstrates that we can emulate \emph{disjunctions} of type constraints, provided that constraints in question can be recovered from some statically computable \emph{witnesses}.
We use a field accessor for a union of extensible records as an example.
\Cref{sec:plugins} shows a practical example of a dependently-typed plugin system type-checkable dynamically at runtime.
There, we see how the combination of the Deferrable constraint pattern~\cite{Kmett:2020ab} and witness manipulation can be useful.
We also discuss the design of the witness of type-level equalities.
Finally, we conclude in \Cref{sec:concl}.

A complete working implementation is available \ifbool{maingo}{in the \href{https://github.com/konn/demotion-paper/tree/master/demotion-examples}{\texttt{demotion-examples}} directory of the support repository~\cite{demotion-repo}}{as the supplementary tarball}.

\subsection{Preliminaries}
In this paper, we use the standard method of \emph{singletons}~\cite{Eisenberg:2012} to simulate dependent types in Haskell.
Briefly, a singleton type \hask{Sing a} of a type-level value \hask{a} is the unique type that has the same structure as \hask{a}, on which we can pattern-match to retrieve its exact shape.
\hask{Sing a} can be identified with a type \hask{a} but exists at the expression-level.
For the detail of singleton-based programming, we refer readers to Eisenberg--Weirich~\cite{Eisenberg:2012} and Lindley--McBride~\cite{10.1145/2503778.2503786}.
In what follows, we assume the following API:

\begin{code}
type family Sing :: k -> Type

class Known a where -- SingI in singletons
  sing :: Sing a
withKnown :: Sing a -> (Known a => r) -> r

data SomeSing k where
  MkSomeSing :: Sing (a :: k) -> SomeSing k
class HasSing k where -- SingKind in singletons
  type Demoted k 
  demote :: Sing (a :: k) -> Demoted k
  promote :: Demoted k -> SomeSing k

withPromoted :: HasSing k
  => Demoted k
  -> (forall x. Sing (x :: k) -> r) -> r

type FromJust :: ErrorMessage -> Maybe a -> a
type family FromJust err may where 
  FromJust err 'Nothing = TypeError err
  FromJust _ ('Just a)  = a
\end{code}
We also assume the following interface for the singletons for GHC's type-level natural numbers:
\begin{code}
type instance Sing = (SNat :: Nat -> Type)
sNat :: KnownNat n => SNat n

withKnownNat :: SNat n -> (KnownNat n => r) -> r
(%+) :: SNat n -> SNat m -> SNat (n + m)
sMod :: SNat n -> SNat m -> SNat (n `Mod` m)
\end{code}
\hask{Nat} is the kind of GHC's built-in type-level natural numbers, and \hask{KnownNat n} is the constraint stating that the compiler knows the concrete value of \hask{n} statically at the compile-time.
These are provided in the \texttt{GHC.TypeNats} module.

We use the following convention:
\begin{enumerate}
\item We prefix singleton types with the capital \hask{S}: e.g.\ \hask{SNat n} is the type of a singleton of \hask{n :: Nat}.
\item For a type-level function we use small \hask{s} as a prefix for singletonised expression-level function: \hask{sMod} is the singletonised version of \hask{Mod}.
\item For operators, we prefix with \haskinline{%}: \haskinline{(%+)} is the singleton for the type-level \hask{(+)}.
\end{enumerate}
\end{document}
