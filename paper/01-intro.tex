%!TeX root = demotion-paper.tex
\documentclass[demotion-paper.tex]{subfiles}
\begin{document}
\section{Introduction}
Since Haskell had been given a Promotion~\cite{Yorgey:2012}, using Haskell with dependent types is a joy.
It's not only a joy: it gives us a neat language to express invariants of program with type-level constraints.

brabra...

Brouwer--Heyting--Kolmogorov (BHK) interpretation of intuitionistic logic: proofs of some proposition are identified with witnessing ``functions''.
For example, proof of $P \land Q$ is given by the pair of proofs $\tau: P$ and $\upsilon: Q$; $P \to Q$ is given by a function taking a proof of $P$ and returns $Q$, ---and so on.
This conception of intuitionistic logic is later made precise as Curry-Howard correspondence and Kleene realisability.

\subsection{Preliminaries}
Throughout this paper, we use the standard method of \emph{singleton}~\cite{Eisenberg:2012} to simulate dependent types in Haskell.
In particular, we assume the following API\footnote{A working implementation is available at \texttt{https://github.com/XXXXX} (made available after blind reviews; for now, we provide the identical contents as the supplementary tarball.)}:

\begin{code}
type family Sing :: k -> Type

class Known a where
  sing :: Sing a
withKnown :: Sing a -> (Known a => r) -> r

data SomeSing k where
  MkSomeSing :: Sing (a :: k) -> SomeSing k
class HasSing k where
  type Demoted k 
  demote :: Sing (a :: k) -> Demoted k
  promote :: Demoted k -> SomeSing k

withPromoted :: HasSing k
  => Demoted k
  -> (forall x. Sing (x :: k) -> r) -> r

type FromJust :: ErrorMessage -> Maybe a -> a
type family FromJust err may where 
  FromJust err 'Nothing = TypeError err
  FromJust _ ('Just a)  = a

type instance Sing = (SNat :: Nat -> Type)
sNat :: KnownNat n => SNat n

withKnownNat :: SNat n -> (KnownNat n => r) -> r
(%+) :: SNat n -> SNat m -> SNat (n + m)
sMod :: SNat n -> SNat m -> SNat (n `Mod` m)
\end{code}

For the detail of singleton-based programming, we refer readers to Eisenberg--Weirich~\cite{Eisenberg:2012} and Lindley--McBride~\cite{10.1145/2503778.2503786}.

We use the following convention:
\begin{enumerate}
\item We prefix with the capital \hask{S} for singleton type: for example, \hask{SNat} is the singleton type of the kind \hask{Nat}.
\item For type-level function we use small \hask{s} as a prefix for singletonised  expression-level function: \hask{sMod} is the singletonised version of \hask{Mod}.
\item For operators, we prefix with \hask{%}: \hask{(%+)} is a singleton for type-level \hask{+}.
\end{enumerate}
\end{document}
