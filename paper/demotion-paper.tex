\documentclass[sigplan,review]{acmart}
\usepackage{subfiles}
\usepackage[gray,prologue]{xcolor}
\usepackage{cleveref}
\usepackage{braket}
\usepackage{tikz}
\usetikzlibrary{matrix,decorations,arrows,calc,trees}
\usepackage{tcolorbox}
\tcbuselibrary{listings,xparse}
\usepackage[draft]{fixme}
\usepackage{enumitem}

\newcommand{\R}{\mathbb{R}}
\newcommand{\N}{\mathbb{N}}
\newcommand{\rem}[2]{{\overline{#1}}^{#2}}
\newcommand{\X}{\boldsymbol{X}}
\newcommand{\Y}{\boldsymbol{Y}}
\newcommand{\Rseries}{\mathord{\R\llbracket\X\rrbracket}}
\DeclareMathOperator{\NonVan}{NV}
\DeclareMathOperator{\RF}{RF}
\newcommand{\Tower}{\mathord{\mathrm{Tower}}}
\newcommand{\smoothto}{\xrightarrow{C^\infty}}

% Syntax
\definecolor{ltblue}{rgb}{0,0.4,0.4}
\definecolor{dkblue}{rgb}{0,0.1,0.6}
\definecolor{dkgreen}{rgb}{0,0.35,0}
\definecolor{dkviolet}{rgb}{0.3,0,0.5}
\definecolor{dkred}{rgb}{0.5,0,0}
\definecolor{comment}{HTML}{444444}
\definecolor{keywd}{HTML}{8D00ED}
\definecolor{types}{HTML}{1F7B2F}
\definecolor{str}{HTML}{4070a0}
\definecolor{code-background}{gray}{0.8}
\definecolor{pragma}{HTML}{372A78}
\definecolor{num}{HTML}{40a070}
\definecolor{symb}{HTML}{000000}
\newcommand\symbmath[1]{{\ensuremath{\color{symb}#1}}}
\newcommand\typemath[1]{{\ensuremath{\ul{#1}}}}
\newcommand{\moncompose}{\mathbin{>\mkern-6mu=\mkern-6mu>}}
\newcommand\colmath[2]{{\ensuremath{\color{#1}#2}}}
\def\ul#1{{\underline{\bfseries #1}}}

\let\textt=\texttt
% Since I mistype for so many times! :-P

\usepackage{minted}
\setminted{style=borland}

\newminted[code]{haskell}{%
mathescape,linenos,numbersep=5pt,frame=lines,framesep=2mm,%
}
\newminted[repl]{haskell}{%
mathescape,numbersep=5pt,frame=lines,framesep=2mm,%
}
\VerbatimFootnotes
\DeclareTotalTCBox{\hask}{v}{%
tcbox raise base,box align=base,verbatim,colback=lightgray,colframe=gray%
}{\mintinline[fontsize=\SMALL]{haskell}{#1}}
\let\haskinline=\hask


% \setcopyright{acmcopyright}
% \copyrightyear{2021}
% \acmYear{2021}
% \acmDOI{10.1145/1122445.1122456}

%% These commands are for a PROCEEDINGS abstract or paper.
\acmConference[Haskell '21]{Haskell Symposium '21: the 2021 ACM SIGPLAN Symposium on Haskell}{August 26--27, 2021}{Virtual}
\acmBooktitle{Haskell '21: the 2021 ACM SIGPLAN Symposium on Haskell, August 26--27, 2021, Virtual}
\acmPrice{15.00}
\acmISBN{978-1-4503-XXXX-X/18/06}

%%\acmSubmissionID{123-A56-BU3}

%%\citestyle{acmauthoryear}

\begin{document}

%%
%% The "title" command has an optional parameter,
%% allowing the author to define a "short title" to be used in page headers.
\title[Functional Pearl: Witness Me]{Functional Pearl: Witness Me --- Constructive Arguments Must Be Guided with Concrete Witness}
%%
%% The "author" command and its associated commands are used to define
%% the authors and their affiliations.
%% Of note is the shared affiliation of the first two authors, and the
%% "authornote" and "authornotemark" commands
%% used to denote shared contribution to the research.

% \author{Hiromi Ishii}
% \email{h-ishii@math.tsukuba.ac.jp}
% \affiliation{%
%   \institution{DeepFlow, Inc.}
%   \streetaddress{3-16-40}
%   \city{Fujimi-shi Tsuruse nishi}
%   \state{Saitama prefecture}
%   \country{Japan}
%   \postcode{354-0026}
% }

% \renewcommand{\shortauthors}{Hiromi Ishii}

\begin{abstract}
  
\end{abstract}

\begin{CCSXML}
<ccs2012>
   <concept>
       <concept_id>10003752.10003790.10003796</concept_id>
       <concept_desc>Theory of computation~Constructive mathematics</concept_desc>
       <concept_significance>500</concept_significance>
       </concept>
   <concept>
       <concept_id>10003752.10003790.10011740</concept_id>
       <concept_desc>Theory of computation~Type theory</concept_desc>
       <concept_significance>500</concept_significance>
       </concept>
   <concept>
       <concept_id>10003752.10003790.10002990</concept_id>
       <concept_desc>Theory of computation~Logic and verification</concept_desc>
       <concept_significance>250</concept_significance>
       </concept>
 </ccs2012>
\end{CCSXML}

\ccsdesc[500]{Theory of computation~Constructive mathematics}
\ccsdesc[500]{Theory of computation~Type theory}
\ccsdesc[250]{Theory of computation~Logic and verification}


%%
%% Keywords. The author(s) should pick words that accurately describe
%% the work being presented. Separate the keywords with commas.
\keywords{Haskell,dependent types,promotion,demotion,singletons,polymorphism,kinds,invariants,type-level programming}

\begin{abstract}
  Beloved Curry--Howard correspondence tells that types are intuitionistic propositions, and in intuitionistic logic school, a proof of proposition can be seen as some kind of a construction, or \emph{witness}, conveying the information of the proposition.
  We demonstrate how this point of view is useful as the guiding principle for developing dependently-typed programs.
\end{abstract}

%%
%% This command processes the author and affiliation and title
%% information and builds the first part of the formatted document.
\maketitle

\subfile{01-intro}
\subfile{02-toy-example}
\subfile{03-disjunction}
\subfile{04-plugins}
\subfile{05-concl}

\begin{acks}
Techniques presented in this paper were found while developing software at my company.
I thank my colleagues for fruitful discussions.
\end{acks}

\bibliographystyle{ACM-Reference-Format}
\bibliography{references.bib}

\appendix

% \section{Appendix?}

\end{document}
%%
%% End of file `sample-lualatex.tex'.
\endinput