%!TeX root = demotion-paper.tex
\documentclass[demotion-paper.tex]{subfiles}
\begin{document}
\section{Conclusions}
\label{sec:concl}
We demonstrated how we could use the constructive point of view, paying attention to \emph{witnesses}, as a useful guiding principle in designing embedded dependent type-systems in Haskell.
As a concrete example, we have demonstrated:
\begin{enumerate}
  \item Type-level arguments \emph{witnessing} evaluation paths in a type-family enables us to safely write corresponding singletonised function much easier afterwards.
  \item Disjunctions of type constraints can be emulated if it is recoverable from \emph{witnesses} that is statically computable at type-level.
  \item Combining \emph{witnesses} with \hask{Deferrable} class, we can implement a dependently-typed plugin system, which can be dynamically type-checked at \emph{runtime}.
  \item It is convenient to provide a unified witness type for extensionally equivalent, but not definitionally equivalent constraints; we took type-level equalities as an example.
\end{enumerate}
To summarise, what we demonstrated in this paper is not a single \emph{method}, but an insightful way of thinking when one designs dependently typed programs.

\subsection{Related and future works}
The examples of this paper incorporates many existing works on dependent types in Haskell: those include type-families~\cite{Kiselyov:2010aa}, data-type promotion~\cite{Yorgey:2012}, singletons~\cite{Eisenberg:2012}, Implicit Configuration~\cite{Kiselyov:2004aa}, to name a few.
The Hasochism paper~\cite{10.1145/2503778.2503786} contains many examples of dependently-typed programming and obstacles in Haskel.
We demonstrated how useful the constructive point of view is when we write dependently-typed codes incorporating these prominent methods.
There is a minor difference in the direction of interest, though.
In dependently-typed programming in Haskell, the method to \emph{promote} functions and data-types is widely discussed~\cite{Yorgey:2012,Eisenberg:2012,10.1145/2503778.2503786}.
On the other hand, many examples in the present paper arise when one tries to \emph{demote} type-level hacks down to the expression-level.
Such needs of demotion arise when one wants to resolve dependently-typed constraints at runtime, as we saw in \Cref{sec:plugins}.
The demotion-based approach has another advantage compared to the promotion-based one: the behaviours of macros for deriving definitions are much predictable.
For example, in \texttt{singletons}~\cite{singletons} package, there are plenty of macro-generated type-families with names with seemingly random suffixes.
This makes, for example, writing type-checker plugins to work with them rather tedious.

We have used user-defined witnesses carrying instance information, such as singletons and equality witness.
If once coherent explicit dictionary applications~\cite{Winant:2018wu} get implemented in GHC, we will be able to directly treat instance dictionaries as another kind of witnesses.

Since the contribution of this paper is a general point of view, there is much room for exploration of synergy with other methods with dependent types.
For example, the method of Ghosts of Departed Proofs~\cite{Noonan:2018aa} shares the witness-aware spirit with the present paper.
It suggests aggressive uses of phantom types to achieve various levels of type-safety.
It can be interesting to explore the synergy of such methods with our examples.
For example, we can use the \hask{Reified} class with default instances instead of \hask{Given} to switch selection strategies based on phantom type parameters.

The compilation performance matters much when one tries to apply involved type-level hacks in the industry and there are many possibilities for exploration in this direction.
For example, suppose we promote container types, such as trees, and provide basic construction as a type-function, rather than data-constructors.
Then, as the current GHC doesn't come with the ability to inline type-level terms, it can take so much time to normalise such type-level constructions when they appear repeatedly.
In such cases, the compiler plugins to expand type families at compile time can help to improve the compilation time; but this is only a partial workaround and more investigations must be taken.
\end{document}